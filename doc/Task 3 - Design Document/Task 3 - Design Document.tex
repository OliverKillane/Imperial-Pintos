\documentclass{report}
	\title{50007.3 - Laboratory - Task 3 - Design Document}
	\author{Robbie Buxton, Jordan Hall, Bartłomiej Cieślar and Oliver Killane}
	\date{01/12/21}
\usepackage[a4paper, total={6in, 8in}]{geometry}
\usepackage[dvipsnames]{xcolor}
\usepackage{graphicx, amssymb, amsfonts, amsmath, xcolor, listings, tcolorbox, enumerate, hyperref}

\graphicspath{{image/}}

\definecolor{codebackdrop}{gray}{0.9}
\definecolor{commentgreen}{rgb}{0,0.6,0}
\lstset{
	inputpath=../../src/vm,
	commentstyle=\color{commentgreen},
	keywordstyle=\color{blue}, 
	backgroundcolor=\color{codebackdrop}, 
	basicstyle=\footnotesize,
	frame=single,
	numbers=left,
	stepnumber=1,
	showstringspaces=false,
	breaklines=true,
	postbreak=\mbox{\textcolor{red}{$\hookrightarrow$}\space}
}

% COMMON TOOLS:
\newcommand{\question}[1]{\textit{#1} \ }
\newcommand{\sidenote}[2]{\begin{tcolorbox}[title=#1]#2\end{tcolorbox}}
\newcommand{\bullpara}[2]{\item \textbf{#1} \ #2}
\newcommand{\keyword}[1]{\textbf{#1}}

% SYNTAX HIGHLIGHTING:
\newcommand{\fun}[1]{\textcolor{Emerald}{\textbf{#1}}}
\newcommand{\file}[1]{\textcolor{YellowGreen}{\textbf{#1}}}
\newcommand{\struct}[1]{\textcolor{orange}{\textbf{#1}}}
\newcommand{\var}[1]{\textcolor{RoyalPurple}{\textbf{#1}}}
\newcommand{\const}[1]{\textcolor{BrickRed}{\textbf{#1}}}

% CODE LISTINGS:
% \pintoscode{startline}{endline}{title}{file}
\newcommand{\pintoscode}[4]{\lstinputlisting[language=C, firstline = #1, firstnumber = #1, lastline = #2, title = #3]{#4}}

% \pintoscode{startline}{endline}{file}
\newcommand{\pintosfile}[3]{\pintoscode{#1}{#2}{\file{#3}}{#3}}

% \codeline{language}{line}{file}
\newcommand{\codeline}[3]{\lstinputlisting[language=#1, firstline = #2, lastline = #2]{#3}}

\newcommand{\centerimage}[2]{\begin{center}
	\includegraphics[#1]{#2}
\end{center}}

% Shorter bullet point and numbered lists
\newcommand{\compitem}[1]{\begin{itemize}\setlength\itemsep{-0.1em}#1\end{itemize}}
\newcommand{\compenum}[1]{\begin{enumerate}\setlength\itemsep{-0.1em}#1\end{enumerate}}

\begin{document}
	\maketitle

	\section*{Group}
		\question{Fill in the names and email addresses of your group members.}
		\begin{center}
			\begin{tabular}{r l}
				\textbf{Name} & \textbf{Email} \\
				Bartłomiej Cieślar & bc1520@ic.ac.uk \\
				Jordan Hall & jh4020@ic.ac.uk \\
				Oliver Killane & ok220@ic.ac.uk \\
				Robert Buxton & rb419@ic.ac.uk \\
			\end{tabular}
		\end{center}
	
	\section*{Preliminaries}
		\subsection*{Latex Formatting}
		To make this document more readable, we have a consistent style:
		\subsubsection*{D4 (5 marks)}
		\question{This is the question's text\dots
		\\
		\\ \dots it can go over several lines!}
		\\ We also have syntax highlighting for:
		\begin{itemize}
			\item Functions such as \fun{schedule} or \fun{init\_thread}.
			\item Variables such as \var{num\_ready\_threads} or 
			\var{thread\_mlfqs}.
			\item Structures of type definitions such as \struct{thread} or 
			\struct{ready\_queue}.
			\item Constants and values set in define such as 
			\const{BINARY\_POINT} and \const{PRI\_MAX}.
			\item Files such as \file{thread.c} and \file{fixed-point.h}.
		\end{itemize}
		We also include code directly from the repository, so the contents, line 
		numbers and file titles all match their respective places.
		\pintoscode{98}{110}{\file{syscall.c}}{/../userprog/syscall.c}
		This makes our document quicker and easier to read and mark.

		\subsection*{Interesting things we've used for the project}
		During the project we have set up our own tools to increase the speed of
		development. Some of the tools we used are listed below:

			\subsubsection*{Continuous Integration}
			We have set up Gitlab's CI to run the test suite on a custom Docker 
			image designed for PintOS. This runs tests for each of the parts we 
			have implemented. If any of these tests fail, then the branch cannot
			merge into the master branch. This ensures that any "quick changes" 
			made to the master branch do not break the code base.
			
			\subsubsection*{Linter}
			In order to enforce a single code-style across all of our changes, 
			we extended the makefile to include a built-in linter. The CI 
			pipeline also includes a linting pass over all files that we have 
			edited throughout the project. The linter we decided to use was 
			the checkpatch linter, which we sourced from the linux codebase 
			\href{https://github.com/torvalds/linux/blob/master/scripts
			/checkpatch.pl}{here}. It follows the linux kernel style.
			
			\subsubsection*{Formatting Scripts}
			Paired with the linter, we have used the clang-formatter and a 
			custom python script (see pintos\_16/reformat.py) to reformat the 
			codebase to conform with the linux kernel style. 

			\subsubsection*{Live Code Sharing}
			For tasks 2 and 3 we have adopted a more collaborative approach. 
			By having pairs work on a problem simultaneously using the 
			VSCode liveshare functionality, we found that more bugs were caught 
			early,our code quality improved and we could make design 
			changing decisions quicker.

	\section*{Page Table/Frame Management}
		\subsection*{Data Structures}
			\subsubsection*{A1 (2 mark)}
				\question{Copy here the declaration of each new or changed 
				`struct' or `struct' member, global or static variable, 
				`typedef', or enumeration that relates to your supplemental 
				page table and frame table. Identify the purpose of each in 
				roughly 25 words.}
				
				\centerimage{width=\textwidth}{page table entry.png}
				% -------------------------------------------------------------
				% ---BARTEK WHAT DOES THIS MEAN IT DOESN'T MAKE SENSE----------
				% -----------------------142 - 145 ----------------------------
				% -------------------------------------------------------------
				We decided not to use a supplemental page table because we 
				deemed it an unnecessary waste of memory. We found out that, by
				using a type of Huffmann encoding, we can place the necessary 
				data in the paged-out entries of the page table. 
				Storing the data about paged-out entries in the page table also has 
				an added benefit of the page table entry being fetched into cache during
				the TLB miss, which means that the data for it can be fetched directly from the L1 cache. 
				This increases the performance of the page fault handler significantly. \\
				In the case that the page is paged-in, we do not 
				need to worry about it since we are not going to page fault on 
				them during normal operation. When a pte is paged-out, it has 
				several possible states:
				\compitem{
					\bullpara{Swapped entry}
						{\\In this case, with the remaining bits that are not 
						used for the encoding, we store a swap frame id which 
						is going to be used when paging the entry in. Because we
						can only store up to 29 bits, we are limited in terms of
						the maximum swap size with the upper limit of 1TB. This 
						should still be plenty compared to the maximum of 64MB 
						of RAM available in the system.}
					\bullpara{Lazy-zeroed entry}
						{\\In this case, the first remaining bit that is not 
						used for the encoding determines if the zeroed page 
						should be writable or not. The remaining bits are used 
						for auxiliary data that the thread can use when setting 
						up the process. We use this space in our setup in order 
						to keep the maximum mapping size for the lazy loading of
						frames.}
					\bullpara{Mmaped or lazy-loading entry}
						{\\ In this case, we have 30 bits remaining for storing 
						the pointer to the structs that manage the mmaped or 
						lazy-loaded entries. The ISO specification ensures that 
						all malloced structs are allocated at address with the 
						same alignment as the struct. Because both structs are 
						4 bytes aligned, their pointers will also be 4 bytes 
						aligned. This means we can discard the leading 2 bits 
						in those pointers and fit them in the remaining space 
						of the pte.To determine if the pointer is to an mmap 
						page (\struct{user\_mmap} struct) or a lazily loaded 
						page (\struct{lazy\_page} struct) we dereference the 
						pointer as a boolean. The first field of both structs 
						is a boolean, false for \struct{user\_mmap} and true for 
						\struct{lazy\_page}. We can then cast the pointer 
						accordingly. This adds at most 4 bytes due to the 
						alignment, but still outweighs having to allocate a 
						supplemental page table.}
				}
				It is important to note that getting a frame's PTE is atomic.
				This is because it means that there are no data races in 
				aquiring a frame's associated PTE and the data relating to that
				entry.
				
				% frame struct (fte)
				\pintosfile{11}{22}{frame.c}
				\begin{center}
					\begin{tabular}{l p{10cm}}
						\vspace*{2mm}
						\var{used\_elem} & This \struct{list\_elem} is used to 
						handle the frame table entry within the 
						\struct{used\_queue}. \\ \vspace*{2mm}

						\var{is\_swappable}  & This boolean determines if the 
						frame is to be written to swap space (stack pages and 
						writeable lazy-loaded) with \file{swap.\{c,h\}} 
						functions, or back to the file system (mmap system 
						pages). \\ \vspace*{2mm}

						\const{union}  & This is a union of the data required 
						for both swappable and mmaped pages.  \\ \vspace*{2mm}

						\var{shared\_mmap}  & A pointer to the 
						\struct{shared\_mmap} struct that owns the frame. The 
						file, offset, 
						length and other data required for reading from and to disk as well 
						as the page table entries using the \struct{shared\_mmap} are stored
						here. (For mmapped pages)
						\\ \vspace*{2mm}

						\var{pd} and \var{vpage}  & The page directory and virtual page 
						within that directory using the frame. (For swappable pages) 
						\\ \vspace*{2mm}
					\end{tabular}
				\end{center}
				
				% frame table (*ftes)
				\pintosfile{38}{38}{frame.c}
				An array of all frame table entry (\struct{fte}) structs. 
				This is allocated once from the kernel pool in the initialisation of the 
				frame system \fun{frame\_init}. The size of the array is equal to the number
				of pages in the user pool.
				\\
				\\ Frames are accessed using kernel addresses (This is very fast and 
				simple to lookup).
				\[\text{frame table index} = \cfrac{\text{frame kernel address} - 
				\text{user pool base kernel address}}{\text{\const{PGSIZE} - 
				page size is currently 4096 bytes}}\]
				
				% user_base
				\pintosfile{41}{41}{frame.c}
				This is used for the translations of kernel page addresses to frame 
				table indexes. It is the base address of the user pool from palloc.
				
				% used_queue
				\pintosfile{53}{53}{frame.c}
				This is a circular queue of frames currently being used, that are 
				unlocked (can	be evicted). Our second chance page replacement algorithm 
				runs on this queue.

				% used_queue lock
				\pintosfile{44}{44}{frame.c}
				This synchronizes access to the \struct{used\_queue}.

				% unlocked_frames semaphore
				\pintosfile{50}{50}{frame.c}
				In general, we represent the page-locked (page-pinned) pages as those
				not in the \var{used\_queue} and not free in the palloc system.
				This semaphore is used to ensure that free or unlocked frames are available in \fun{frame\_get}.
				If there are no free frames available from palloc and there
				are no unlocked frames in the \struct{used\_queue} that can be page replaced,
				downing this semaphore will result in a wait until one is available. This
				also means that, in this instance, all pages are page-locked which is extremely rare and does not
				happen for too long, but it is still something that we need to consider.
				

		\subsection*{Algorithms}
			\subsubsection*{A2 (2 mark)}
				\question{Describe your code for finding the frame (if any) or 
				other location that contains the data of a given page.} 
				\\
				When a frame is lazy-loaded or lazy-zeroed we do not need 
				to locate the page because it has not been set up yet. \\
				Otherwise the frame is mmaped or swapped, so there are two 
				possible options: the frame can be locked or it can be 
				determined not to be in memory. \\
				Frames are locked with \fun{frame\_lock\_swappable()} and 
				\fun{frame\_lock\_mmaped()}. Here, we determine if the frame can
				successfully be locked by checking the owner pointer inside the 
				frame. If the owner supplied to the frame locking function 
				equals the frame's owner, then it can be safely locked.
				Otherwise, there are 2 possible options for what happens 
				next (in case of a page-fault for example):
				\compitem{
					\bullpara{If our entry is swappable}{
						Here if locking the page fails, it is certain that
						it will stay inside of the swap system. This means that
						the swap entry id from the pte can be used to locate it.
					}
					\bullpara{If our entry is mmaped}{
						Here if locking the page fails, it is uncertain
						if it will stay paged-in when accessing is required. 
						However, because it is possible attempt to lock the
						frame there must be a \struct{user\_mmap} pointer
						to the shared page that was previously contained in that fram.
						Therefore the \struct{shared\_mmap} that represents it can be locked
						and all the necessary actions performed while holding that lock.
					}
				}
				Because fetching the data about the pte is atomic,
				it is certain that the data inside ptes is accessable without a 
				race condition. This data is then used for determining the type 
				of page that is being delt with.

			\subsubsection*{A3 (2 mark)}
				\question{How have you implemented sharing of read only pages?}
				\\ The sharing of read-only pages is implimented using the same
				system as for the sharing the mmaping pages.
				\\ When opening the mmap in read only mode, the writes to the 
				file are blocked and the pages are marked as read-only.
				\\ When loading the writable pages, where bytes must be read from 
				the executable file, they are lazy-loaded separately. Access is 
				denyed to writing to the file before the lazy-loading has happened.
			
		\subsection*{Synchronization}
			\subsubsection*{A4 (2 mark)}
				\question{When two user processes both need a new frame at 
				the same time, how are races avoided? You should consider both 
				when there are and are not free frames available in memory.}
				
				\compitem{
					\item The semaphore \var{unlocked\_frames} is used 
					to ensure that there are available frames to get from 
					palloc, or to evict.
					\item The lock \var{used\_queue\_lock} is used to 
					synchronize access to the \var{used\_queue}.
				}
				\compenum {
					\bullpara{Down \var{unlocked\_frames} semaphore}{
						\\ If there are no free or unlocked pages in used\_queue
						and can be evicted, the process waits until one is present. 
						Frame locking using \fun{frame\_lock\_mmaped} and 
						\fun{frame\_lock\_swappable} also downs this semaphore.
						If there is only one frame available then access to it
						for frame locking is prevented.
					}
					\bullpara{Poll palloc for free frames}{
						\\ If \fun{palloc\_get\_page(\const{PAL\_USER})} returns
						a page, we can return this. It is not in 
						\var{used\_queue} so it is frame locked and 
						synchronized on the palloc system. If \const{NULL} the 
						only available frames are unlocked frames so the second 
						chance algorithm is run.
					}
					\bullpara{Get access to \var{used\_queue} through \var{used\_queue\_lock}}{
						\\ A lock is used to enforce exclusive access to the 
						\var{used\_queue} in every instance of frame locking,
						unlocking, getting and freeing. This prevents race 
						conditions on frames being used because before
						use it is locked.
					} 
					\bullpara{Evict frame}{
						\\ The second chance algorithm selects a frame,
						removes it from the \var{used\_queue}, and evicts it.
						The relevant eviction function (\fun{mmap\_frame\_evict()} 
						for mmap pages and \fun{swap\_page\_evict()} for swappable)
						is used.
					}
					\bullpara{Release \var{used\_queue}}{
						\\ The lock on the \var{used\_queue} is released.
					}
					\bullpara{Return frame}{
						\\ The frame is not in the \var{used\_queue} and therefore
						is frame locked and cannot be evicted.
						This means it is safe to use for loading.
					}
				}
				If two processes call \fun{frame\_get()} at the same time and 
				there are at least 2 frames free or unlocked, both processes 
				will be able to \fun{sema\_down()} the \var{unlocked\_frames} 
				semaphore without waiting.
				\\
				\\ Otherwise a process may have to wait until some frames are 
				free or unlocked, although this is extremely rare.
				\\
				\\ The processes access palloc, which is synchronized internally.
				If either get a frame, then it is returned.
				\\ 
				\\ Otherwise if there are not enough free frames, then a 
				processe will need to get frames by page replacement. 
				The \var{used\_queue\_lock} ensures only one process can evict 
				by page replacement at once, ensuring synchronisation. Other 
				processes can still evict by many by other means such as
				process exit and munmap.
			
		\subsection*{Rationale}
			\subsubsection*{A5 (2 mark)}
				\question{Why did you choose the data structure(s) that you did for 
				representing the supplemental page table and frame table?}

				\textbf{Design of the frame table:}
				\\ We reduced our design down to two main options.
				\\ \begin{minipage}[t]{0.45\textwidth}
					\centerline{Array based, single allocation table}
					\textcolor{green}{
						\compitem{
							\item It requires only one memory allocation.
							\item It has very fast lookup times using pointer
							arithmetic.
							\item It never fail at allocating a new frame.
						}
					}
					\textcolor{red}{
						\compitem{
							\item All frame table entries are allocated, even 
							when very few are being used leading to higher 
							memory usage.
						}
					}
				\end{minipage}
				\hfill
				\begin{minipage}[t]{0.45\textwidth}
					\centerline{Hashmap with dynamically allocated entries}
					\textcolor{green}{
						\compitem{
							\item IOnly uses memory for frames being used.
						}
					}
					\textcolor{Red}{
						\compitem{
							\item It has higher maximum memory usage when all 
							frames are in memory, must also store hashmap.
							\item It has a slower lookup time as it requires 
							hashing and potentially iterating through buckets 
							in the hashmap.
							\item The new frames require multiple allocations.  
							One is needed for the frame itself and one for
							the hashmap.
							\item The frame allocations can fail.
							\item The hashmap insertions can require malloc and 
							failure degrades hashmap performance.
							\item The freeing of frames is slower and 
							requires a call to \fun{free} for the table entry.
						}
					}
				\end{minipage}
				In the end an array based, singly allocated frame table was 
				chosen for its better performance. We considered the main 
				drawback, higher memory usage when there 
				are very few used frames, to be negligable. This is because for
				the default in tests, 367 pages in user pool, it only uses 
				$\approx 2.5$ kernel pages out of 367.
				\\
				\\ Furthermore kernel memory usage only becomes an issue when 
				running many processes or using many pages. Examples include
				mmaps, file I/O and when there are lots of child processes. 
				Under these conditions our design will be much faster and 
				potentially use even less memory if the number of used user 
				frames is large enough.
				
				\textbf{Lack of a Supplemental Page Table}
				As previosuly mentioned, we decided to encode necessary 
				information for non-present pages directly into the page table.
				
	\section*{Paging To And From Disk}
		\subsection*{Data Structures}
			\subsubsection*{B1 (1 mark)}
				\question{Copy here the declaration of each new or changed `struct' or
				`struct' member, global or static variable, `typedef', or enumeration 
				that relates to your swap table. Identify the purpose of each in roughly
				25 words.}
				
				\pintosfile{10}{10}{swap.c}
				The helper value for writing pages to swap slots where each swap 
				slot contains \const{BLOCKS\_PER\_PAGE} blocks.

				% swapid_t
				\pintosfile{9}{9}{swap.h}
				The type definition to make swap index distinct.
				
				% is_free_tree
				\pintosfile{12}{12}{swap.c}
					We decided to use an interval tree for determining if a 
					swap slot is empty. For a swap space of size \var{n} swaps 
					(where \var{n} is twice the the number of real swap slots
					rounded up to the nearest power of 2), the bitmap for the 
					interval tree is of size $2*\var{n}-1$. The interval tree 
					allows us to find a free page in $log(\var{n})$ steps.
				
				% is_writeable
				\pintosfile{13}{13}{swap.c}
					This is a simple bitmap for checking if the swapped frame 
					is writable or not. It is indexed in constant time.

				% swap_lock
				\pintosfile{15}{15}{swap.c}
					
					\var{swap\_lock} synchronizes the \var{is\_free\_tree}. 
					Also synchronizes reading and writing to the swap block 
					device. The synchronisation over the swap block device 
					separate to the rest of the file system allows the two 
					devices to be written to/read from at the same time.

		\subsection*{Algorithms}
			\subsubsection*{B2 (2 mark)}
				\question{When a frame is required but none is free, some frame 
				must be evicted. Describe your code for choosing a frame to 
				evict.}
				\\
				\\ When we want to aquire a frame in \fun{frame\_get()} and none
				are free, we decide on a page to evict using a second-chance 
				algorithm via a circular queue implemented with a linked list. 
				When we evict a frame, we check the front of the queue. 
				If the accessed bit or the disjunction of accessed bits in the 
				case of shared pages for the frame are 1, then we reset that 
				bit, the accessed bit, to 0 and put it to the back of the queue.
				If the accessed bit for the frame was 0, then we evict the frame
				and return the new page.
			
			\subsubsection*{B3 (2 mark)}
				\question{When a process P obtains a frame that was previously 
				used by a process Q, how do you adjust the page directory of
				process Q (and any other data structures) to reflect the frame 
				Q no longer has?}
				\\
				\\ In the frame evition we acquire the \var{used\_queue\_lock}.
				Then we lock the frame (Q's frame) that is being evicted.
				We also acquire the relevant lock for eviction of the frame 
				prior to releasing the \var{used\_queue\_lock}. This is so we 
				can allow further evictions, without race conditions on writing 
				back and reading from the swap/shared mmap's page in it file.
				
				\compenum {
					\item Before writing Q's page to swap/disk, 
					the relevant page table entries as marked as evicted.
					\item \compitem{
						\bullpara{Swappable}
						{Write swap slot ID to Q's page table entry.}
						\bullpara{Mmap}
						{Write \struct{user\_mmap} pointers to all relevant page
						table entries}}
					\item As a result Q or any other process using the frame 
						will page fault on access, and wait on the 
						\var{used\_queue\_lock} before loading the data back.
						This means there is no race condition on accessing
						pages as they are evicted.
					\item Next we give the locked frame to process P. Process P 
						can then use \fun{swap\_load()} or \fun{mmap\_load()} to 
						load the required data to the frame. 
					\item First the data is loaded to the frame, then the 
						relevant page table entry set in P. Finally the frame
						is unlocked and can be evicted, and P resumes. We then
						write the data back to swap/file system.
				}
				Page table entries are set in a single store operation which is 
				atomic to avoid race conditions on page table entries in the 
				format specified and explained in A1.

		\subsection*{Synchronization}
			\subsubsection*{B4 (2 mark)}
				\question{Explain how your synchronization design prevents deadlock.
				(You may want to refer to the necessary conditions for deadlock.)}

				The 4 conditions for a deadlock are:
				\compitem{
					\bullpara{Mutual Exclusion}{Each resource is either available or assigned to exactly one process}
					\bullpara{Hold \& Wait}{Process can request resources while it holds other resources}
					\bullpara{No preemptions}{Resources given to a process canot be forcibly revoked or borrowed}
					\bullpara{Circular Wait}{Two or more processes in a circular chain, each waiting for a resource held by the next process}
				}
				The only synchronization methods that fit the first 3 conditions are locks and semaphores used throughout the virtual memory system.
				Those, however, when acquired simultaneously, are always acquired in the same order. If we imagine a graph of all possible
				nestings/chainings that can occur in our system (based on the lock/semaphore acquisition diagrams above), it will be a DAG(Directed Acyclic Graph).
				Indeed, the only synchronization primitives chains/nestins longer than 1 primitive that can happen are:
				\compitem{
					\bullpara{\fun{frame\_get()} evicting a swapped page and paging in a swappable/mmaped page (and unlocking it)}{
						\compenum{
							\item \var{unlocked\_frames}
							\item \var{used\_queue\_lock}
							\item \var{swap\_lock}
						}
					}
					\bullpara{\fun{frame\_get()} evicting an mmaped page and paging in a swappable/mmaped page (and unlocking it)}{
						\compenum{
							\item \var{unlocked\_frames}
							\item \var{used\_queue\_lock}
							\item \var{shared\_mmap$\to$lock}
							\item \var{filesys\_lock}
						}
					}
					\bullpara{either \fun{frame\_lock\_mmaped()} or \fun{frame\_lock\_shared()} locking a page}{
						\compenum{
							\item \var{unlocked\_frames}
							\item \var{used\_queue\_lock}
						}
					}
					\bullpara{either \fun{mmap\_register()} or \fun{mmap\_unregister()} managing the mmapings}{
						\compenum{
							\item \var{mmaps\_lock}
							\item \var{shared\_mmap$\to$lock}
							\item \var{filesys\_lock}
						}
					}
					\bullpara{\fun{mmap\_load()} loading an mmaping}{
						\compenum{
							\item \var{mmaps\_lock}
							\item \var{filesys\_lock}
						}
					}
					\bullpara{\fun{read}/\fun{write} syscall}{
						\compenum{
							\item \var{read\_write\_buffer\_lock}
							\item \var{filesys\_lock}
						}
					}
				}
				Therefore, no deadlock can occur, since no resources can be acquired in a circular chain (by the definition of a DAG).

				\subsection*{The following images are visual aids to our synchronisation system:}
					\begin{minipage}{\textwidth}
						The legend for following diagrams:
						\centerimage{width=\textwidth}{synch legend.png}
					\end{minipage}
					\begin{minipage}{\textwidth}
						Loading a new child process:
						\centerimage{width=\textwidth}{synch load.png}
					\end{minipage}
					\begin{minipage}{\textwidth}
						Mmap reigster and unregister handling:
						\centerimage{width=\textwidth}{synch mmap.png}
					\end{minipage}

					\begin{minipage}{\textwidth}
						Page faulting and getting a new frame from the pool:
						\centerimage{width=\textwidth}{synch page fault.png}
					\end{minipage}

					\begin{minipage}{\textwidth}
						Syscall handling:
						\centerimage{width=\textwidth}{synch syscall.png}
					\end{minipage}

					\begin{minipage}{\textwidth}
						Thread exit handling:
						\centerimage{width=\textwidth}{synch thread exit.png}
					\end{minipage}

				
				
			\subsubsection*{B5 (2 mark)}
				\question{A page fault in process P can cause another process Q's frame to
				be evicted. How do you ensure that Q cannot access or modify the page 
				during the eviction process?)}
				\\
				\\As stated in B3, we flag Q's frame as evicted before writing 
				it to swap. It is important to note that modifying the PTE of 
				Q's page to mark it as evicted is atomic. We only write the Q's
				frame's data after marking it as evicted. This is so no data 
				races can occur between Q's page being set as evicted and Q's 
				attempts to modify or access data in its page.
				
			\subsubsection*{B6 (2 mark)}
				\question{A page fault in process P can cause another process Q's frame
				to be evicted. How do you avoid a race between P evicting Q's frame and 
				Q faulting the page back in?}
				\\
				\\ When evicting, we maintain exclusive access to the 
				\var{used\_queue} for the minimum time possible by acquiring 
				\var{used\_queue\_lock} and passing it to 
				\fun{swap\_frame\_evict()}. This function releases the 
				\var{swap\_lock} after it has finished writing the evicted frame
				to swap. Therefore, if Q attempts to load the page back in 
				from swap, then it must wait until P has finished writing Q's 
				frame to swap. This is because \var{swap\_lock} must also be 
				acquired in \fun{swap\_load()} in order to load the data from 
				swap.

			\subsubsection*{B7 (2 mark)}
				\question{Explain how you handle access to user pages that are not 
				present when a system call is made.}
				\\
				\\We use a system similar to the one we used in task 2.
				We use a \fun{put\_user()} and \fun{get\_user()} functions that
				can put a character to the user page or get it respectively. Then,
				using those functions, we safely copy the data to/from the user's buffer
				to/from a buffer inside of kernel. If the page that we are reading from/writing to
				is present, then those calls will pass through easily. Otherwise,
				one of the following can occur:
				\compitem{
					\bullpara{We successfully page that data in}{
					\\ In that case we return to the \fun{put\_user()} or \fun{get\_user()} call
					and proceed as if that data was paged-in.}
					\bullpara{We fail to page any page in}{
					\\ In that case we inform the \fun{put\_user{}} or \fun{get\_user()} that fetching
					the data has failed.}
				}
				For read and write syscalls,an additional buffer is used for copying the intermediate data to.
				This buffer is shared between all read and write syscalls, since the writes
				to the filesystem are coarsly synchronized anyways and having different buffers for the writes and reads
				could potentially make them fail more often in a higher system congestion situation. The buffer is of course
				synchronized on accesses to it using the \var{read\_write\_buffer\_lock}.
			
		\subsection*{Rationale}
			\subsubsection*{B8 (2 mark)}
				\question{There is an obvious trade-off between parallelism and the 
				complexity of your synchronisation methods. Explain where your design 
				falls along this continuum and why you chose to design it this way.}
				\\
				\\Our implementation primarily focuses on squeezing as much performance and
				as little memory utilization from the hardware as possible. On second place
				stands our synchronization, which we tried to do without significant costs
				to the performance of our system. That means that accesses to the structures
				that were deemed potentially being in a lot of instances had to be either
				synchronized locklessly or with a coarser lock. Usually, that trade-off didn't mean
				that much loss in terms of performance due to other limitations. On top of that,
				we still did our vertical synchronization (over separate structures we access
				throughout execution) as fine-grained as possible, which resulted in a significant
				performance and memory utilization gained while loosing a minimal amount of parallelism.
				\\
				\\In terms of optimizing sychronization, we decided that the page-faulting is the
				most time-critical seciton of our system, so we wanted to make the synchronisation
				there as fine-grained as possible without worrying too much about the other parts of the
				code in which we manage the aspects of the virtual memory. Such parts include setting up
				the executable for a new process, exiting a process, or mmaping a file to user's memory.
				\\
				\\ As far as the general idea of our synchronization goes, wherever an accesses to user's pte is needed,
				it is done atomically to ensure no race-conditions there. A similar thing is the case for the framing system,
				where an for inconsistency between the pte that has just been fetched is allowed. This enforces a need for
				failing to lock a frame. On the other hand, this allows for not having locks for ptes at all, since in all places
				where there are accesses to ptes, any race conditions on them are irrelevant or solved some other way. 
				Then there accesses to the framing system, which are chained with accesses to the mmaping and swapping systems.
				Those need to be chained to prevent potential race conditions between setting the ptes during eviction and accessing those systems.
				
	\section*{Memory Mapped Files}
		\subsection*{Data Structures}
			\subsubsection*{C1 (1 mark)}
				\question{Copy here the declaration of each new or changed `struct' or 
				`struct' member, global or static variable, `typedef', or 
				enumeration that relates to your file mapping table. Identify 
				the purpose of each in roughly 25 words.}
				The mmaping system has 4 major functionalities:
				\compitem{
					\bullpara{Registering a user to a shared mmaped page}{}
					\bullpara{Unregistering a user from a shared mmaped page}{}
					\bullpara{Evicting a shared mmaped page}{}
					\bullpara{Loading in a shared mmaped page}{}
				}
				When registering an mmap for a user, \struct{user\_mmap} is used
				for that task. It represents a handle for the user to a \struct{shared\_mmap}
				it uses. Then, if there is a \struct{shared\_mmap} that matches the mmaping being registered
				on: its length within a page, offset in the file, the inode for that file and writability,
				then the \struct{user\_mmap} is added to it. Otherwise, a new \struct{shared\_mmap} is constructed
				and the \struct{user\_mmap} is added to it.
				\\
				\\ When unregistering an mmap for the user, the pointer to the \struct{user\_mmap} is obtained
				by using the list containing the \var{mmap\_id\_elem}s. Then, the pointer to the specific \struct{shared\_mmap}
				is obtained from the pointer to it in \struct{user\_mmap}. Then, a check is done in a synchronized way on whether
				the \struct{user\_mmap} that is being unregistered is the last one registered for that \struct{shared\_mmap}. If it is not,
				then just \struct{user\_mmap} gets removed from that \struct{shared\_mmap}. Otherwise, \struct{shared\_mmap} is destroyed as well.
				\\
				\\ When paging an mmaping out, first all of the \struct{user\_mmap} that are inside the \var{used\_mmaped\_pages} list have their
				page table entries set to evicted, and only then is the data about that page written back to the file.
				\\
				\\ When paging an mmaping in, the reverse is performed.
				First, the pointer to the \struct{user\_mmap} is obtained
				by using the list containing the \var{mmap\_id\_elem}s. Then, the pointer to the specific \struct{shared\_mmap}
				is obtained from the pointer to it in \struct{user\_mmap}. Subsequently, it is checked if
				the mmaping has not possibly been already loaded in (which is used in conjunction with the frame locking to avoid race conditions).
				Then, the mmaping is written back to the given page. Finally, all of the \struct{user\_mmap} in the \var{used\_mmaped\_pages} have
				their page table entries set to paged in.
				
				% mmaps
				\pintosfile{37}{37}{mmap.c}
				Hash map of all the \struct{shared\_mmap} instances. This is used for 
				sharing the mmapings in \fun{mmap\_register()}.

				% mmaps_lock
				\pintosfile{40}{40}{mmap.c} 
				The lock for the hash map above.
				
				% shared_mmap
				\pintosfile{45}{57}{mmap.c}
				Handles a page sized block of an mmaped file, with a list of all 
				\struct{user\_mmap's} using of it. \\
				\begin{center}
					\begin{tabular}{l p{10cm}}
							\vspace*{2mm}
							\var{mmap\_system\_elem}  & A \struct{hash\_elem} that is the elem for \struct{mmaps} \\ \vspace*{2mm}
							\var{*file}               & A \struct{file} that is the write-back file. This is also used for indenficication \\ \vspace*{2mm}
							\var{file\_offset}        & A \const{off\_t} that is the offset in the file that we are mapping \\ \vspace*{2mm}
							\var{length}              & An \const{int16\_t} that is the length of the chunk of the file that we are mapping \\ \vspace*{2mm}
							\var{writable}            & A \const{bool} that is if the mapping should be writable or not \\ \vspace*{2mm}
							\var{dirty}               & A \const{bool} that is if any of the removed users had their dirty bit set \\ \vspace*{2mm}
							\var{lock}                & A \struct{lock} that is the general lock used for this \struct{shared\_map}\\ \vspace*{2mm}
							\var{used\_mmaped\_pages} & A \struct{list} that is the list of all users of that mmap \\ \vspace*{2mm}
					\end{tabular}
				\end{center}
				
				% user_mmap
				\pintosfile{60}{67}{mmap.c}
				Handles the user's page table entry for an mmaped page. \\
				\begin{center}
					\begin{tabular}{l p{10cm}}
							\vspace*{2mm}
							\var{is\_lazy}            & A boolean used to determine whether this struct is a \struct{lazy\_load} or \struct{user\_mmap}. Always set to \const{false.} \\ \vspace*{2mm}
							\var{mmap\_id\_elem}      & A \struct{list\_elem} used for either the list of pages for an mmaping or the list of executable mmapings. \\ \vspace*{2mm}
							\var{shared\_mmap\_elem}  & A \struct{list\_elem} used for the \var{used\_mmaped\_pages} in \struct{shared\_mmap} \\ \vspace*{2mm}
							\var{*shared\_mmap}       & A \struct{shared\_mmap} that is the manager of the shared page for this mmap \\ \vspace*{2mm}
							\var{*pd}                 & The page directory of the process that owns that \struct{user\_mmap} \\ \vspace*{2mm}
							\var{*vpage}              & A \struct{void} pointer to the user page that this \struct{user\_mmap} manages \\ \vspace*{2mm}
					\end{tabular}
				\end{center}

				% lazy_load
				\pintosfile{8}{13}{lazy.c}
				Struct used to store lazy-loaded pages which have not been loaded from disk yet. Become swappable pages when
				they are loaded into memory.  \\
				\begin{center}
					\begin{tabular}{l p{10cm}}
							\vspace*{2mm}
							\var{is\_lazy}        	& A boolean used to determine if the page is loaded lazily. Should always be set to true. \\ \vspace*{2mm}
							\var{*file}      		& A pointer to a \struct{file} that the page is sourced from. \\ \vspace*{2mm}
							\var{file\_offset}  	& The place in the file where this lazy-loaded page starts. \\ \vspace*{2mm}
							\var{length}       		& The number of bytes of the file to read and store in the page. \\ \vspace*{2mm}
					\end{tabular}
				\end{center}


		\subsection*{Algorithms}
			\subsubsection*{C2 (2 mark)}
				\question{Explain how you determine whether a new file mapping overlaps 
				with any existing segment and how you handle such a case.}
				\\
				\\ This case is handled by checking the type of the pte before registering an mmaped page on it.
				If the type is \const{NOTSET} then it must be true that no page is registered on it.
				This holds true since any sequence of page-ins/page-outs will not change the status of
				the pte from or to \const{NOTSET}; the only place where it happens is when a process is loaded or exits,
				in which case it is not happening at the instance the mmaping is being installed in. Moreover, the accesses
				to the value of pte are atomic, so checking its type does not cause any synchronisation issues.
				\\
				\\ The mmaped pages are registered sequentially for the whole mmaping. If one of them fails, as described above, then
				all mmaped pages are unregistered and the mmaping function returns \const{$-1$}. Otherwise, the malloced list of mmapings
				is assigned to the first empty entry in the vector of mmapings, or pushed back to it if no spot is empty. Then, the
				index for that mmaping is returned. Since mmaped pages reopen the files separately, the original file used for mmaping can be safely closed
				without breaking the mmapings associated with it.
			
			\subsubsection*{C3 (2 mark)}
				\question{Mappings created with "mmap" have similar semantics to those of 
				data demand-paged from executables. How does your code-base take advantage 
				of this?}
				\\
				\\We decided to take advantage of the similarities between mappings created with "mmap" and data demand-paged from executables
				by merging the two systems into a single monolithic system. They are treated the same way when lazy loaded. 
				The only difference is that the executables are not writable, and so when we unregister an mmaping belonging to an executable,
				we do not write-back to the file under any circumstances.
				
				Due to the specifications of ELF files, there needs to be sections of mmaped files that behave in the
				following way:
				\compitem{
					\bullpara{They must be lazy loaded like the rest of the executable}{}
					\bullpara{When we read or write to them for the first time, a page faults occurs as usual}{}
					\bullpara{When a page-fault occurs on a lazy mmaped page, a locked frame is aquired, the contents are copied
					from the file to the frame, set the PTE to the new frame, and unlock the frame making it a normal swappable
					page that can be evicted}{}
				}

				
				% Extend this to talk about the three different mmappings - and why we need lazy mmaps

\end{document}
